\documentclass[11pt,a4paper]{article}
\usepackage[top=1.00in, bottom=1.0in, left=1.1in, right=1.1in]{geometry}
\renewcommand{\baselinestretch}{1.2}
\usepackage{graphicx}
\usepackage{natbib}
\usepackage{hyperref}

\newenvironment{smitemize}{
\begin{itemize}
  \setlength{\itemsep}{1pt}
  \setlength{\parskip}{0pt}
  \setlength{\parsep}{0pt}}
{\end{itemize}
}

\def\labelitemi{--}

\begin{document}
\bibliographystyle{/Users/Lizzie/Documents/EndnoteRelated/Bibtex/styles/amnat}
\renewcommand{\refname}{\CHead{}}
% \title{Forecasting Vitis}:

\title{Coordinated phenological responses to environmental cues across a community of temperate forest plants}
\date{13 May 2017}
\author{Flynn \& Wolkovich}
\maketitle

{\bf Outline}


\begin{enumerate}

\item Introduction
\begin{enumerate}
\item Phenology is super important, to ecosystems and trophic levels and stuff ... and also to coexistence within one trophic level---the temporal niche
\item Climate change has increased research (and interest in phenology) -- lots of change seen, but as data across species has piled up, somewhat conflicting and contrasting results (Fu, chilling etc.). Touch on invasions here? 
\item Accurate predictions clearly require a fuller understanding of the interacting environmental cues that drive phenology within a community. We list out the cues (and say temporal niche again).
\item We somewhere say that we expected a trade-off between cues!
\item Here we study these cues in some forests.
\end{enumerate}

\item Results 
\begin{enumerate}
\item Strong main effects of forcing, chilling and photoperiod. Strong interactions of forcing x chilling; some other important interactions
\item Species varied in their cues, but all species has all 3 cues. Shrubs generally had weaker cues than trees, though this was not always consistent.
\item Responses varied quantitatively depending on whether BB or LO was considered.
\end{enumerate}

\item Discussion
\begin{enumerate}
\item Species and community responses
\begin{enumerate}
\item Spring phenology and community assembly: Our results suggest species have paced budburst and leafout due to a mix of thee major environmental cues. In contrast to any trade-off between cues, we found coordinated responses across the community. This means ...
\item All species had all cues. This jives with many models of phenology (cite Chuine etc.) but diverges from some stated assertions and from a number of chamber studies. Why? Because the cues all interact so if you clip branches once chilling is met or perhaps do the Weinberger thing, you might not see chilling and/or photoperiod cues.
\end{enumerate}
\item Comparing cues
\begin{enumerate}
\item Big effects of forcing and chilling, and photoperiod
\item Chillng at two levels gave similar responses
\item Forcing and chilling offset (cite Fu, Cook etc.) -- implications for climate change 
\item Site had smaller effects, but chilling x site was substantial (raises concerns for single site chilling studies?)
\end{enumerate}
\item Comparing phases: BB vs. LO
\item Conclusions: Tie it all back to climate change
\begin{enumerate}
\item Tie to exotic/non-native work? Tie to pioneer/climax work? 
\item This means predicting any one species' response to climate change may not be simple, let alone predicting community-level responses. 
\end{enumerate}
\end{enumerate}

\end{enumerate}



{\bf Methods:}\\
Total observations, total clippings, total individual trees .... \\
We focused on woody plants, because they can be manipulated in chambers and should be in most direct competition with one another (compared to herbs versus shrubs or such) ....\\



% \newpage
\noindent \emph{Some references}\\

\begin{footnotesize}
{\def\section*#1{}
\bibliography{/Users/Lizzie/Documents/EndnoteRelated/Bibtex/LizzieMainMinimal}
}
\end{footnotesize}

\end{document}

