\documentclass[11pt,a4paper]{letter}
\usepackage[top=1.00in, bottom=1.0in, left=1.1in, right=1.1in]{geometry}
\usepackage{graphicx}

%\signature{}

\begin{document}
\begin{letter}{}
\includegraphics[width=0.4\textwidth]{/Users/Lizzie/Documents/Professional/images/letterhead/arnold/AASmLogo2colr.jpg}

\opening{Dear Dr. XXX:}

\noindent Please consider our revised manuscript `XX' as a \emph{XXX} for \emph{New Phytologist.} 
\vspace{1.5ex}\\
Our manuscript examines how to buffer many crops from major climate change impacts through improved use of existing diversity. We believe this piece will have broad interest to your readers. For researchers studying climate change impacts on crops it would suggest a major new approach to crop modeling, and relatedly alter how researchers collect and share data---with implications for growers and breeders. Beyond agricultural research, we believe the piece will inspire interest and research in both ecology and evolution---by highlighting the role of plant traits in improving climate change research and by the need to better understand the underlying genetic drivers of this diversity. 
\vspace{1.5ex}\\
Comments from three reviewers have greatly improved this manuscript. We have strengthened links from our case study of winegrapes to other crops, especially in the introduction, where we now more fully discuss the value of focusing on existing diversity. Additionally, we have increased our discussion of how feasible many of our suggested approaches are: adding information on where wine regulations may hamper changing varieties and discussing avenues to promote data sharing. We have added two new figures: one shows predicted climate change in major wine growing regions with a layer displaying information on estimated variety regulations in different countries, the other is a conceptual figure to better explain how a trait framework could help identify promising varieties for different regions. Additionally we have added a number of requested references, as well as other references we found in addressing reviewers' comments, and made numerous small changes to address concerns, and keep our word count reasonable. 
\vspace{1.5ex}\\
We have attempted to address all reviewer concerns, but word length considerations and somewhat contrasting requests by reviewers (e.g., Reviewer \#2 wanted substantial changes to the section `Expand diversity research to commercial vineyards,' while Reviewer \#3 specifically praised it as `one of the stronger sections of paper') meant that we were not always able to comprehensively address all comments. We have tried to balance giving more information where possible (sometimes adding longer discussions of complex issues such as wine quality and yield, or European regulations on wine, to the Supplementary Materials) with maintaining a coherent, focused narrative. 
\vspace{1.5ex}\\
We feel the new submission is much improved and detail our changes in the following pages (note that reviewer comments are in \emph{italics}, while our responses are in regular text). This manuscript is 4,600 words long with a 102 word summary and 6 figures. We hope that you will find it suitable for publication in \emph{Nature Climate Change}, and look forward to hearing from you.
\\
\\\vspace{-1ex}\\
\noindent Sincerely,\\

 \includegraphics[width=0.3\textwidth]{/Users/Lizzie/Documents/Professional/Vitas/Signatures/SignatureLizzieSm.png} \\

\noindent Elizabeth M Wolkovich

\end{letter}
\end{document}



