\documentclass[11pt,a4paper]{article}
\usepackage[top=1.00in, bottom=1.0in, left=1.1in, right=1.1in]{geometry}
\usepackage{graphicx}
\usepackage{natbib}
\usepackage{amsmath}
\usepackage{hyperref}
\usepackage{todonotes}

\setlength\parindent{0pt}

\begin{document}
\bibliographystyle{/Users/Lizzie/Documents/EndnoteRelated/Bibtex/styles/amnat}

{\bf Editor's comments:} \\

\emph{The main issues raised by Reviewer 2 were: (1) the need for more detailed analyses and
nuanced discussion, particularly with respect to understanding what factors might account for
the inter-specific differences reported in your study; and, (2) concerns about the inclusion
of data in the analyses for species where cuttings often did not break bud.  I agree with the
reviewer that this is a concern and that additional experiments are likely needed to
understand why this happened, and that inclusion of such data in the overall analysis may
have had unexpected effects on the overall results.}\\

We appreciate the concerns of Reviewer 2 and have worked to address them. In regards to (1) we agree that understanding the interspecific variation is a very important goal. We have updated our discussion to highlight some possibilities for what could drive these differences and are  actively working on this area of research through data collection on major traits at our sites, including data on frost sensitivity (see also response to Reviewer 1). \\

We feel our paper is a major contribution in two important ways: (a) to our knowledge this is the first multi-species study to experimentally manipulate all three cues: forcing, photoperiod and chilling, as previous studies have generally manipulated only two cues at once (as Reviewer 3 notes the use of artificial chilling is an important addition), and (b) providing a community perspective in how these cues vary across species. In reading the reviews we realized we erred in not fully discussing (b) and outlining how our a community perspective may help understand what drives variation between species. We have now added this to the discussion (lines 319-334):

\begin{quote}
Our community-level findings may help build on our understanding of what factors ultimately shape each species' mix of cues for budburst and leafout. Recent work has addressed this issue by examining how attributes such native/invasive status, climatic range, or climatic history predict cues \citep[e.g.,][]{Laube2015,zohner2017}. Building on these insights will require improved understanding of phenology's role in defining a species niche and controlling its inclusion in a community. For example, assembly theory suggests early-active species could out-compete later-active species through priority effects, which would produce communities where all species leafout early. When this is not the case (as in our data and many other systems) trade-offs may explain variation in phenology at the community-level \citep{Chesson:1997dz}. In temperate forests one dominant hypothesis for this trade-off is that early-active species should also have traits that allow them to survive or avoid tissue loss to frost \citep{Sakai:1987aa} while later-active species would need traits that allow them to be competitive for resources after other species have already had access to resource pools (e.g., soil nutrients or light). Testing these hypotheses requires matched trait and phenology data with a focus on careful measures of frost sensitivity (e.g., the minimum temperatures tissues can experience without damage) and traits related to competition (e.g., resource uptake metrics or growth rates under varied nutrient and competitive environment regimes).
\end{quote}

In regards to concern (2), relating to our non-leafout rates, we can understand the reviewer's concerns. While we would have liked our non-leafout rates to be zero for all species, we believe this is difficult or impossible to achieve in a species-rich study, such as the one we present here. Indeed, in discussions with colleagues we have found our rates to be the same or better as most other studies (more detail on this below). Many studies, however, do not publish their non-leafout rates so we can understand if the reviewer was surprised by some of our results. We have now updated our manuscript with analyses of the non-budburst and the non-leafout data (lines 142-147), provided additional tables (S2-S4) to make these rates absolutely clear and added to the methods (lines 115-116) a note of how our hierarchical model structure is designed to handle data with varying sample sizes across species (due to non-leafout or other factors). Finally, we re-ran our models on a subset of the data with higher rates of budburst and leafout success and found similar results. \\

\emph{While more supportive, Reviewer 3 questioned a number of your interpretations, and raised the
possibility that your cuttings had not received sufficient chilling in some cases, with
consequences for interpreting relative responses to the forcing treatments.}\\

We appreciate Reviewer 3's concerns and have worked to address them (see below). In particular we have added more information on chilling in our system and another year of data on experimental chilling. These additions, we believe, provide strong evidence that our experimental chilling treatments did provide sufficient chilling for most of our species. \\

\emph{My own view is that your study addresses an important area of plant biology, and provides
interesting data on the interactive effects of the three forcing factors.  However, the
reviewers have raised some important points – addressing them will require provision of new
data (e.g. assessing the importance of pre-cutting chilling, and how it influences the
outcomes of your results, as well as working out why bud break was so poor in some species).}\\

We are glad to hear the editor thinks our paper provides interesting new data on an important topic and we believe our revisions further bolster our findings: we now provide new data from an experiment conducted in the following year and new analyses of our non-budburst and non-leafout data, which we hope address the editor's and reviewers' concerns. \\

\emph{If you do decide to re-submit your study to New Phytologist, please ensure that you follow
the Author Guidelines with respect to formatting of the Summary and References.  For the
latter, check every reference to ensure that you provide the correct details for journal
names (being consistent with New Phytologist requirements) and that species names are in
italics font and use correct capital letters for the genus component (e.g line 383).  Please
also insert a comma after “studies” on line 19, and replace the comma with a semi-colon in
“phenology,”.  On line 73, change “asses’ to “assess”. On line 180, I wondered why you used
the term “community” here; consider removing it.  Finally, I would prefer the use of “By
contrast, “ on line 22 of page 3 (“In contrast” is usually followed by “to” or “with”).}\\

We thank the editor for these corrections and have made all of them. We corrected the Heide reference mentioned and issues we found in about half a dozen other references and we removed the word `community' (indeed, while we meant `community' in respect to `community ecology' on line 180 we see that the use of it was confusing and unclear in this context; as it discussed much more clearly later in the main text we have removed it from this part of the discussion), as well as making all the other smaller changes requested. \\

{\bf Reviewer 1 -- comments:} \\

\emph{I think this is a valuable contribution to the phenology literature. I like the experimental approach, and think it provides greater insight into the phenological responses than most other studies.  The appropriate literature is cited and the tables and figures are good.\\
I've made some minor editorial suggestions on the PDF.\\
David Inouye}\\

We are glad to hear Dr. Inouye thinks this is a valuable contribution to the phenology literature and we appreciated his comments to improve the manuscript. We have inputted all corrections from the PDF including fixing a spelling error, improving sentence structure, removing the noted split infinitives and improving the caption of Figure 2 for clarity about the color coding (we carried this correction through to Figure 3 and three similar supplemental figures). \\

Additionally in the PDF, one comment came up in the section in which we discuss cues across the community of species:\\

\emph{Any correlation with frost sensitivity?}\\

This an excellent question and one we are certainly interested in. Currently, however, we do not have any good metrics of frost sensitivity. A graduate student in the lab has started working on measurements for approximately 10 of the species studied in this paper so we hope to have some answers to this question in the coming years.\\

{\bf Reviewer 2 comments:} \\

\emph{The paper is exceptionally well written and easy to follow. I think that the paper could be
publishable in its current form in some outlets; however, I have two reservations about the
suitability of the paper for publication in a high-impact journal. My strongest reservation
is that the authors did not do much to make sense of differences among species. If the entire
point of the paper is to demonstrate that differences among species exists, then I'm not sure
we have learned very much. I believe that most scientists in this field are already convinced
that differences exist among species and that different species respond to different cues. One barrier to learning more about differences among groups is that the paper does not try to quantify many of the observations gleaned from the figures. In lines 192-194 the authors seem
to claim that a correlation exists at the species level between the effect of photoperiod and
forcing but it isn't quantified. It would be easy to do so. The species-level responses could be correlated against many variables that could yield insights.}\\

We appreciate the reviewer's comments on our writing style and understand his/her concerns regarding the need to understand what drives species-level variation in the cues. As we mention in the introduction (lines 11-16):

\begin{quote}
Understanding this variation has been the goal of much recent work \citep{Rutishauser:2008fu,Laube2015,donnelly2017,zohner2017}, with research focusing on two major linked aims: (1) identifying and quantifying the environmental cues that drive spring phenology (i.e., budburst and leafout), and (2) identifying what drives variation in cues between different species.
\end{quote}
We think this is very important area of study, however, we feel it also requires careful, quantitative work on how the cues vary across species. As much debate has discussed whether all species should have all cues and how these cues interact \citep{Korner:2010,Chuine:xb}, we feel our paper is an important contribution in giving robust estimates of species responses to all three cues as well as their interactive effects, and tests how the cues vary across space. In particular our paper provides an important community perspective and is the only multi-species study to our knowledge that manipulated all cues experimentally in controlled environments. \\

While we completely agree that we could correlate our model-estimated responses to cues to many variables, we see several hurdles to this. First, in trying to carefully examine the cues across species we have developed models that allow us to estimate 14 responses per species, thus any correlations with other variables could easily find spurious correlations without a careful approach to which variables to test. Second, we do not feel we have the data on the major variables currently, which we think would ideally be taken from the same individual trees and sites. \\

We have added to the discussion a review of how we feel our paper contributes in this area, see lines 319-334 (or above in our response to the editor's comments where we quote this new text).\\

We also have added statistical information on the relationship mentioned in the discussion (previously lines 192-194) to the results (lines 182-185, and corresponding information on the statistical approaches in the methods, lines 138-140):
\begin{quote}
Across species responses to forcing and photoperiod were related for budburst (mean slope of 0.31, CI of 0.15-0.48) and leafout (mean slope of 0.45, CI of 0.26-0.66), whereas responses between forcing and chilling were only weakly related (budburst: mean slope of 0.12, CI of 0.04-0.20; leafout: mean slope of 0.11, CI of 0.04-0.20).
\end{quote}

\emph{I disagree with the author’s statement in lines 203-211 that it is not useful to bin species as sensitive or
insensitive based on Figure 2. There is a tight cluster of species that had fairly small
effects of treatments on budburst in Figure 2A and then there are some with larger effects.
Are these not two groups?} \\

We agree that there is some separation between species in Figure 2A, however, we cannot see any statistical way to bin these species: any divide we have tried to make ends up including one or more species from the other cluster. For example, \emph{Fraxinus nigra} may be part of the smaller-effect cluster but it overlaps in photoperiod responses with many of the species that had larger effects. Without obvious and clear ways to form bins results using such approaches can easily find spurious correlations \citep[e.g.,][]{gelmanbinning}, which is why we suggest a focus on treating the data as continuous. \\

\emph{Presumably the differences among species are adaptive; they allow
species to survive across wide environmental ranges and occupy different niches.  Responses
to environmental cues provide species’ with phenotypic plasticity.  Can this explain how a
species like Nyssa sylvatica can occur is a very wide range of environment (Florida to Maine)
and Acer saccharum cannot?}\\

We completely agree with the reviewer that these are important questions. Our focus here, however, is not on explaining the range extent of these species and our species were not selected for this area of research, though we agree that our findings would have relevance to this question. \\

\emph{My second reservation is with how he experiment was conducted. In the supplemental material,
the authors report that 20.2\% of cuttings did not break bud. Some species had 50\% failure.
The magnitude of the experimental effect casts doubt on the results.  Is it meaningful to
quantify the time to budbreak in Fagus grandifolia and Acer saccharum when 50\% of cuttings
did not break bud at all?  Can this just be ignored in the analysis?  I don’t think so. What
happened to these cuttings?  Did they fail to conduct enough water and desiccate, or did the
tissues appear healthy but just not break bud? I am not an expert on handling cuttings, but I
wonder if the authors would have had better success if they had taken cuttings earlier in the
dormant season and kept them in a moderate chilling environment before applying the
treatments. If I were conducting the experiment, I would consider this a learning experience
and try to improve my system for acquiring and handling cuttings.  Given, the very strong
experimental effect, I consider the results to be tentative rather than definitive. I am not
sure how these problems can be overcome.}\\

We understand the reviewer's concerns and we have worked to address them in several ways. First, we originally gave only information on non-leafout in the Supporting Information, thus the 20.2\% the reviewer mentions is actually for cuttings that did not leaf out. Our non-budburst rate was only 9.8\% and budburst success was 80\% or higher for all species except \emph{Viburnum lantanoides} (79\%), \emph{Acer saccharum} (78\%) and \emph{Quercus alba} (65\%). We now present these numbers in the supplement's text and have added a supplemental table (S2), which gives budburst and leafout success per species. \\

Second, we contacted a number of colleagues who have done similar experiments \citep[colleagues contacted included authors on the following paper][]{Caffarra:2011aa,Basler:2012aa,Polgar:2014aa,vitasseclippings,zohner2016ncc} to inquire about common rates of cuttings that do not budburst or leafout. The colleagues that had studied budburst looked at fewer species but reported rates of 79-100\% budburst success; more had looked at leafout (across many species) and reported leafout success from 40-100\%, depending on the species, with many species in the 40-50\% range. Some colleagues also reported data for some species for which they did not publish results because leafout success was only 6.5\% (or in some treatments zero percent). \\

We found that many papers reporting results from cutting experiments did not report budburst or leafout success, which may explain in part why these numbers were suprising. But our numbers for budburst and leafout success are in line with or higher than published estimates we found as well: \citet{Basler:2012aa} published budburst success rates of 79-98\%, depending on species, \citet{laube2014gcb} published rates from 10-100\% across species and treatments. Based on these numbers we do not believe our budburst and leafout success rates out outside of what is common for these experiments.\\

Budburst success is known to vary across species for several reasons including chilling \citep{laube2014gcb}, bud position or health (e.g., lower lateral buds may not flush always and buds can die over the winter). For leafout, variation is expected to be due to non-structural carbohydrates (NSC), which vary across species and individuals, and may sometimes be too low for a plant to reach full leafout without additional NSC supply (Y. Vitasse, \emph{pers. comm.}). While researchers have experimented with augmenting the water in flasks in these experiments most suggest changing water often (as we did) is the best practice for robust, comparable estimates across species.\\

We also do not believe our results should be considered tentative for several reasons. First, we specifically selected our hierarchical modeling approach as it is designed to handle variation in sample size. We neglected to mention this in our original submission and now note it on lines 115-116. Second, we now present an analysis of budburst and leafout success and found only very weak relationships with treatments. Third, we re-ran our days to budburst and days to leafout models removing the five species with the highest rates budburst and/or leafout failure and found our main results unchanged. Given our high sample size overall and careful modeling approach, we believe our results are robust. We present an overview of these new results in the main text (lines 142-147):

\begin{quote}
Across all treatments, 9.8\% of the cuttings did not break bud, while an additional 10.4\% did not reach the leafout stage following budburst. Variation was highest due to species identity, but removal of the five species with the lowest  budburst or leafout success did not qualitatively affect the results and quantitatively most estimates changed by less than 10\%. (See \emph{Budburst and leafout success}, Tables S2-S4 and Fig. S2-S3 in the Supporting Information.)
\end{quote}

We also provide supplemental figures of the results (S2-S3). \\

\emph{Additional comments:\\
L14: It was not clear to me at first what was meant by leafout. Maybe say: vegetative
budburst and subsequent leaf development (leafout).\\
L103: I’m not sure that ‘bias’ is the right word here, perhaps ‘effect’ of chamber of flask
position.\\
L108: Leafout if defined here. It would be useful to define it when it is first mentioned.\\
L125:  Should read: We validated that our model code ...}\\

All these changes have been made.\\

\emph{L178-179: The term ‘complex and non-linear’ is vague to me, particular the non-linear part.
Does this mean that it is difficult to extrapolate current trends along a linear trajectory
(e.g., budburst advances 2 days/decade)?  Maybe complex is descriptive enough on its own or
perhaps define what you mean by nonlinear and decide if it adds to the description.}\\

This is a good point. We explain the non-linear aspect later in this same paragraph (i.e., you cannot assume a simple linear response to warming temperatures given the prevalence of other cues) so we removed it from this sentence.\\

\emph{L223: Do you mean that chilling increases or does it accumulate?}\\

Thank you for catching this, we have changed it to `accumulates.'  \\


{\bf Reviewer 3 comments:} \\


\emph{In your manuscript you present results from a climate chamber experiment which investigates
budburst and leaf-out timing under varying chilling, forcing and photoperiod treatments.
The manuscript thus provides new and interesting aspects to an ongoing debate on the
importance of the three major phenological cues as well as on the predictability of climate
change impacts on spring phenology. While the results mainly back up results from similar
previous experiments (although the discussion somewhat overstates minor differences), the
aspect of using artificial chilling is a novel approach and thus absolutely noteworthy.
I think the topic is timely and of broad interest to the climate change community. The
experiment obviously was designed and rolled out with care. The manuscript is well
structured, well written, easy to follow, and the figures are fine.}\\

We thank the reviewer for his/her comments and have worked to maintain these positive attributes of the manuscript in its revised form. \\

\emph{There are some minor comments that might improve the manuscript (below). I only have three
major comments:\\
1.      At several points, you state that your results contradict earlier experiments, which
I actually can ́t see. More precisely, you might discuss the fact that none of the branches
you used had received an amount of chilling typical for their origin. The branches were
harvested in January, and some received additional 30 days of chilling, thus mimicking a
winter lasting until February. Although I regrettably did not visit Boston yet, I ́m very sure
that a usual Boston or Quebeck winter lasts much longer. Thus, my assumption would be that no
branch was fully chilled when entering the forcing/photoperiod treatments. It has been
suggested earlier that a response to photoperiod might be restricted to not fully chilled
individuals. I thus assume that the responses to photoperiod in all species that you find
might be due to a general lack of chilling. This would rather back up previous experiments
than contradict them.
In this context, I ́d suggest to add a figure chilling vs. photoperiod (like Figure 2 or
Figure S4).}\\

This is an interesting point and highlighted for us multiple ways to improve the manuscript. First, we agree this is an important possibility and have added it to lines 234-236: ``thus, one possible reason for our contrasting findings could be that none of cuttings experienced full chilling in our experimental design...".\\

Second, we did design the two chilling treatments to attempt to provide more than sufficient chilling based on data from the two sites, though we did not show this well before. We have now updated Table S7 to show the field chilling experienced, the experimental chilling \emph{and} the chilling the accumulated until 1 April and 1 May at each field site. Estimates from the chilling hours and Utah models both suggest our experimental chilling treatments yielded more chilling than the full season of field chilling and estimates of Chill portions suggest similar chilling between a full winter and the experimental chilling. \\

Third, we have added additional data from a chilling experiment the following year that looked again at different chill temperatures and at shorter periods of chilling (i.e., 16 days versus 32 instead of 30 days as done in the main experiment). We again found very small differences between different temperatures of chilling and we found that doubling the experimental chilling led to only a 33\% increase in the effect, suggesting many of the species may have been approaching full chilling after limited time in the field and limited additional experimental chilling. These new results are now detailed in \emph{Effects of chilling at 16 and 32 days} and Figure S9 in the supplement and referenced on lines 239 and 285-287 in the main text. Finally we have added the requested figures (S7-S8) which show responses to the two chilling treatments versus responses to warming and photoperiod for budburst and leafout (an additional 8 panels). \\

\emph{2.      I read the discussion on different cues leading to budburst and leaf-out with
interest, but do not think that the arguments are convincing enough. On average, you report a
stronger effect of forcing and a stronger effect of photoperiod on leaf-out than on budburst
(Fig. 1). If we assume that between budburst and leaf-out biosynthesis takes place, then
wouldn ́t we expect that longer days and higher temperatures would simply enhance
photosynthesis and thus advance growth? In this sense, are photoperiod and temperature “cues”
(that need to be sensed) or rather limiting factors after budburst?}\\

This is a good point and we have added it where the reviewer suggested (see comments below). Lines 302-305 now read:

\begin{quote}
Such differences may be because budburst and leafout represent fundamentally different responses \citep{Basler:2014aa}: budburst is cued by forcing and photoperiod, whereas leafout generally requires biosynthesis, thus forcing and photoperiod may act more as limiting factors than cues for leafout.  
\end{quote}

\emph{3.  While I really appreciated the idea to use artificial chilling, this also means that
the branches spent more time under unfavorable, artificial conditions. While your results
clearly show that possible negative effects from this are outweighed by advancing chilling
effects, this yet might influence your outcomes, which you should clearly state and discuss.}\\

We completely agree with this point. We have added a new sentence specifically about chilling (lines 239-241), which reads: ``Experimental chilling, however, is highly artificial and fails to replicate the daily, hourly and finer temporal variation in temperature that plants experience in the field as they accumulate chilling \citep{Erez:1988,Luedeling:2009},'' and we have adjusted our discussion in a later paragraph to more harshly point out the limits of experiments such as the one presented here (lines 261-266):

\begin{quote}
The drawback of this approach, however, is that the design is much more artificial in its climate and, given the extreme treatments, may be less relevant for near-term projections and difficult to robustly extrapolate to future conditions. Such designs may be more useful for rough estimates of longer-term predictions of phenological responses with climate change and/or for use in parameterizing process-based models, which often use a mix of results from observations and experiments.
\end{quote}

\emph{Minor comments:\\
p.2 (3): Please add average effect sizes per cue. While reading the summary, I had the false
impression that photoperiod was of real big importance in comparison to the others, which is
not the case.}\\

Done.\\

\emph{p.2 (4): You might add the problem that also climate change predictions that are reliable
with respect to interacting winter chilling, spring warming and photoperiod are hard to
obtain.}\\

This is an excellent suggestion. Unfortunately we are at the word limit for the summary so we have added this point to the conclusions (lines 342-344): ``This is an especially difficult task given that climate change projections that could be used to robustly estimate future forcing and chilling are difficult to obtain (e.g., many projections do not provide daily estimates).''\\

\emph{Line 145-146: Does this sentence make sense? I don ́t get it.}\\

This is a good point, we have re-worked the sentence and believe its meaning is now clearer. \\

\emph{Line 171: Should read shorter days?}\\

Thank you for this, we have made the correction. \\

\emph{Line 194ff: Actually, I can ́t see this from the figure. I ́d suggest that you either provide
statistical evidence for this claim, or remove the sentence.}\\

We have now added these statistics to the results (lines 182-185): ``Across species responses to forcing and photoperiod were related for budburst (mean slope of 0.31, CI of 0.15-0.48) and leafout (mean slope of 0.45, CI of 0.26-0.66), whereas responses between forcing and chilling were only weakly related (budburst: mean slope of 0.12, CI of 0.04-0.20; leafout: mean slope of 0.11, CI of 0.04-0.20).''\\

\emph{Line 216: I think you should add the fact about not fully chilled branches (see above) here.}\\

Thank you, as noted above we have discussed this point here. \\

\emph{Line 256: I think this point needs a bit more elaboration. There is plenty of literature and
discussion (e.g. your citations in the Supplement) arguing on effective temperatures and
complex chilling responses to temperature.}\\

We completely agree that this was a surprising finding and have adjusted the sentence (lines 280-284). In particular, we have added that our two chilling temperatures fall within a range often modelled as accumulating the same (maximum) chilling units \citep[for example, see][]{harrington2015}, thus it may be only for higher temperatures where we should expect a differing response. We also have added new data from an experiment the following year, which again found similar responses across the two chilling temperatures and we reference that as well. \\

\emph{Line 273: I think you should add the point on different biosynthesis rates after budburst,
and thus expected advances with photoperiod and temperature (see above) here.}\\

This is a good point and as noted above we have added this to lines 302-305. \\

\newpage
\bibliography{/Users/Lizzie/Documents/git/projects/treegarden/budexperiments/docs/ms/danlib.bib}

\end{document}
