\documentclass[11pt,a4paper]{article}
\usepackage[top=1.00in, bottom=1.0in, left=1.1in, right=1.1in]{geometry}
\usepackage{graphicx}
\usepackage{natbib}
\usepackage[export]{adjustbox}

% http://www.nature.com/nclimate/authors/gta/ed-process/index.html

% Researchers should supply a brief paragraph stating the interest to a broad scientific readership, address and contact details, title, a fully referenced summary paragraph, and a list of the references cited in the summary paragraph. Additional material can be included as a separate file if needed.

\begin{document}

\noindent \includegraphics[width=0.4\textwidth, right]{/Users/Lizzie/Documents/Professional/images/letterhead/arnold/AASmLogo2colr.jpg}
\vspace{1ex}\\

\noindent Dear Dr. Brown: %Check?
\vspace{1.5ex}\\
\noindent Please consider our manuscript `Temperature and photoperiod drive spring phenology across all species in a temperate forest community' as a \emph{Letter} for \emph{Nature Climate Change.} 
\vspace{1.5ex}\\
Plant phenology plays a crucial role in ecosystem processes and is one of the most reported indicators of climate change \citep{Cleland:2007aa,piao2017,sippel2016}. Yet as the wealth of observational data highlighting this rapid advance in phenology has increased, research has uncovered variation in these shifts across space, time and species \citep{Rutishauser:2008fu,Wolkovich:2012aa,fu2015}. Understanding this variation has led to a number of studies and debates \citep[e.g.,][]{Korner:2010,Chuine:xb} about how the major cues known to underlie phenology---spring forcing temperatures, winter chilling temperatures, and photoperiod---vary across species, and whether they may interact, which would make predictions more complex and non-linear. Observational studies have highlighted that a simple model of temperature forcing cannot predict the observed variation \citep{Rutishauser:2008fu,fu2015,carter2017}, but have been hampered from further insights because the three major cues generally covary in nature. Advances in our understanding therefore require an experimental approach that manipulates all three cues across a community of species. 
\vspace{1.5ex}\\
Here we present results of a full-factorial experiment manipulating all three cues (spring forcing temperatures, photoperiod, and intensity of winter chilling) across 28 woody species and from two North American forests at two latitudes. Using state of the art Bayesian hierarchical models we were able to estimate responses to each cue, and to the interaction of cues, for all studied species---as well as estimate and overall response. Contrary to hypotheses \citep{Korner:2010} and recent work \citep[using methods that do not manipulate all cues,][]{zohner2016ncc}, we found all species responded to all cues. Responses to photoperiod and forcing temperature were related among species and showed no evidence that some species could be categorized as insensitive to any cue, despite recent efforts to create such binary categories \citep{laube2014gcb,zohner2016ncc,donnelly2017}. Chilling exerted a strong effect on phenology and interacted importantly with forcing. Our results suggest that predicting the spring phenology of communities will be difficult as all species we studied could have complex, non-linear responses to future warming \citep{Chuine:1999aa}. 
\vspace{1.5ex}\\
We have suggested three possible reviewers (see comments section of online submission system). Both authors substantially contributed to this work and approved of this version for submission. The manuscript is approximately 2,080 words with 184 word preface, and two figures. It is not under consideration elsewhere. We hope that you will find it suitable for publication in \emph{Nature Climate Change}, and look forward to hearing from you.
\vspace{1.5ex}\\
Thank you for your consideration.
\vspace{1.5ex}\\
\noindent Sincerely,\\

 \includegraphics[width=0.4\textwidth]{/Users/Lizzie/Documents/Professional/Vitas/Signatures/SignatureLizzieSm.png} \\
%\noindent Elizabeth M Wolkovich (on behalf of my co-authors)
\newpage
\noindent {\bf References:}
\vspace{-5ex}
\bibliographystyle{/Users/Lizzie/Documents/EndnoteRelated/Bibtex/styles/naturemag}
\renewcommand{\refname}{\CHead{}}
\bibliography{/Users/Lizzie/Documents/git/projects/treegarden/budexperiments/docs/ms/danlib}

\end{document}

Understanding the cues that control spring leafout in forest communities is critical to accurate predictions of future growing seasons, plant communities, and a suite of related ecosystem services, including carbon sequestration. As such a growing body of literature has ...
Accurate predictions of spring plant phenology with continued climate change are critical for robust projections of growing seasons, plant communities and a suite of ecosystem services. 